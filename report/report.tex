\documentclass[10pt,twocolumn,letterpaper]{article}

\usepackage{cvpr}
\usepackage{times}
\usepackage{epsfig}
\usepackage{graphicx}
\usepackage{amsmath}
\usepackage{amssymb}

% Include other packages here, before hyperref.

% If you comment hyperref and then uncomment it, you should delete
% egpaper.aux before re-running latex.  (Or just hit 'q' on the first latex
% run, let it finish, and you should be clear).
\usepackage[breaklinks=true,bookmarks=false]{hyperref}

\cvprfinalcopy % *** Uncomment this line for the final submission

\def\cvprPaperID{****} % *** Enter the CVPR Paper ID here
\def\httilde{\mbox{\tt\raisebox{-.5ex}{\symbol{126}}}}

% Pages are numbered in submission mode, and unnumbered in camera-ready
%\ifcvprfinal\pagestyle{empty}\fi
\setcounter{page}{1}
\begin{document}

%%%%%%%%% TITLE
\title{\LaTeX\ Author Guidelines for CVPR Proceedings}

\author{Tao Du\\
Stanford University\\
{\tt\small taodu@stanford.edu}
% For a paper whose authors are all at the same institution,
% omit the following lines up until the closing ``}''.
% Additional authors and addresses can be added with ``\and'',
% just like the second author.
% To save space, use either the email address or home page, not both
}

\maketitle
%\thispagestyle{empty}

%%%%%%%%% ABSTRACT
\begin{abstract}
\end{abstract}

%%%%%%%%% BODY TEXT
\section{Introduction}

%------------------------------------------------------------------------
\section{Related Work}

%------------------------------------------------------------------------
\section{Shape Deformation}

%------------------------------------------------------------------------
\section{Optimization Algorithm}

\subsection{Optimizing homography}

\noindent
Given $n$ pairs of correspondent points $\mathbf{p}_i\in\mathcal{R}^3$ and $\mathbf{q}_i\in\mathcal{R}^2$, $i=1,2,\cdots, n$, we want to fit $\mathbf{M}\in\mathcal{R}^{3\times 4}$ such that
\begin{equation}
\mathbf{q}_i=s_i\mathbf{M}\left[
\begin{array}{c}
\mathbf{p}_i\\
1
\end{array}
\right ]
\end{equation}
or equivalently
\begin{equation}
\begin{split}
&m_{11}p_{i1}+m_{12}p_{i2}+m_{13}p_{i3}+m_{14}\\
=&q_1(m_{31}p_{i1}+m_{32}p_{i2}+m_{33}p_{i3}+m_{34})\\
&m_{21}p_{i1}+m_{22}p_{i2}+m_{23}p_{i3}+m_{24}\\
=&q_2(m_{31}p_{i1}+m_{32}p_{i2}+m_{33}p_{i3}+m_{34})
\end{split}
\end{equation}
As a result, if we reshape $\mathbf{M}$ as a column vector $\mathbf{m}$, we can fit $\mathbf{m}$ by solving the following least square problem:
\begin{equation}
\begin{split}
\min_{\mathbf{m}}&\quad \|\mathbf{Am}\|_2^2\\
s.t.&\quad \|\mathbf{m}\|=1
\end{split}
\end{equation}
Here $\mathbf{A}\in\mathcal{R}^{2n\times 12}$. The optimal $\mathbf{m}$ can be found analytically by SVD.

\subsection{Optimizing intrinsic and extrinsic parameters}

\noindent
Once $\mathbf{M}$ is found, we will decompose $\mathbf{M}=\lambda\mathbf{K[R,t]}$. Here we borrow the idea from Zhang~\cite{zhang2000flexible} to fit both the camera matrix $\mathbf{K}$, the rotation matrix $\mathbf{R}$ and the translation vector $\mathbf{t}$. We use $\mathbf{m}_i$ to represent the $i$-th column in $\mathbf{M}$.

The camera matrix $\mathbf{K}$ is an upper triangular matrix:
\begin{equation}
\mathbf{K}=\left [
\begin{array}{ccc}
f & 0 & x \\
0 & f & y \\
0 & 0 & 1
\end{array}
\right ]
\end{equation}
Note that we assume the camera does not have any distortion, and the aspect ratio is 1. Both assumptions are reasonable: if we allow $\mathbf{K}_{12}\neq 0$, or if we allow $\mathbf{K}_{11}\neq\mathbf{K}_{12}$, they will absorb the effects of rotation and scaling, which should actually be explained by the shape deformation. Once $\mathbf{K}$ is determined, $\mathbf{t}$ can be easily solved by $\mathbf{t} = \mathbf{K}^{-1}\mathbf{m}_4$. So the main challenge is to solve $\mathbf{K}$ and $\mathbf{R}$. If we use $\mathbf{r}_i$ to denote the $i$-th column in $\mathbf{R}$, we will have:
\begin{equation}
\begin{split}
\lambda\mathbf{K}\mathbf{r}_1=&\mathbf{m}_1\\
\lambda\mathbf{K}\mathbf{r}_2=&\mathbf{m}_2\\
\lambda\mathbf{K}\mathbf{r}_3=&\mathbf{m}_3\\
\mathbf{r}_1=&\lambda^{-1}\mathbf{K}^{-1}\mathbf{m}_1\\
\mathbf{r}_2=&\lambda^{-1}\mathbf{K}^{-1}\mathbf{m}_2\\
\mathbf{r}_3=&\lambda^{-1}\mathbf{K}^{-1}\mathbf{m}_3\\
\end{split}
\end{equation}
For any $i\neq j$, we have the constraint $\mathbf{r}_i^\top\mathbf{r}_j=0$:
\begin{equation}
\begin{split}
\mathbf{m}_1^\top\mathbf{K}^{-\top}\mathbf{K}^{-1}\mathbf{m}_2&=0\\
\mathbf{m}_1^\top\mathbf{K}^{-\top}\mathbf{K}^{-1}\mathbf{m}_3&=0\\
\mathbf{m}_2^\top\mathbf{K}^{-\top}\mathbf{K}^{-1}\mathbf{m}_3&=0\\
\end{split}
\end{equation}
Moreover, $\mathbf{r}_i$ has the same length:
\begin{equation}
\begin{split}
\mathbf{m}_1^\top\mathbf{K}^{-\top}\mathbf{K}^{-1}\mathbf{m}_1&=\mathbf{m}_2^\top\mathbf{K}^{-\top}\mathbf{K}^{-1}\mathbf{m}_2\\
\mathbf{m}_2^\top\mathbf{K}^{-\top}\mathbf{K}^{-1}\mathbf{m}_2&=\mathbf{m}_3^\top\mathbf{K}^{-\top}\mathbf{K}^{-1}\mathbf{m}_3\\
\end{split}
\end{equation}
All the constraints contain $\mathbf{K}^{-\top}\mathbf{K}^{-1}$. Note that $\mathbf{K}^{-1}$ is
\begin{equation}
\mathbf{K}^{-1}=\left [
\begin{array}{ccc}
\frac{1}{f} & 0 & -\frac{x}{f} \\
0 & \frac{1}{f} & -\frac{y}{f} \\
0 & 0 & 1
\end{array}
\right ]
\end{equation}
So we can define
\begin{equation}
\begin{split}
\mathbf{B}=&\mathbf{K}^{-\top}\mathbf{K}^{-1}\\
=&\left [
\begin{array}{ccc}
\frac{1}{f} & 0 & 0 \\
0 & \frac{1}{f} & 0 \\
-\frac{x}{f} & -\frac{y}{f} & 1
\end{array}
\right ]
\left [
\begin{array}{ccc}
\frac{1}{f} & 0 & -\frac{x}{f} \\
0 & \frac{1}{f} & -\frac{y}{f} \\
0 & 0 & 1
\end{array}
\right ]\\
=&\left [
\begin{array}{ccc}
\frac{1}{f^2} & 0 & -\frac{x}{f^2} \\
0 & \frac{1}{f^2} & -\frac{y}{f^2} \\
-\frac{x}{f^2} & -\frac{y}{f^2} & \frac{x^2+y^2}{f^2} + 1
\end{array}
\right ]\\
=&\left [
\begin{array}{ccc}
B_{11} & 0 & B_{13} \\
0 & B_{11} & B_{23} \\
B_{13} & B_{23} & B_{33}
\end{array}
\right ]
\end{split}
\end{equation}
Since $B$ is symmetric, it only contains $4$ variables. Now for any $i$ and $j$, we can rewrite the constraints as:
\begin{equation}
\begin{split}
\mathbf{m}_i^\top\mathbf{B}\mathbf{m}_j=&\mathbf{v}_{ij}^\top\mathbf{b}\\
=&\left [
\begin{array}{c}
m_{i1}m_{j1}+m_{i2}m_{j2}\\
m_{i1}m_{j3}+m_{i3}m_{j1}\\
m_{i2}m_{j3}+m_{i3}m_{j2}\\
m_{i3}m_{j3}
\end{array}
\right ]^\top
\left [
\begin{array}{c}
B_{11} \\
B_{13} \\
B_{23} \\
B_{33}
\end{array}
\right ]\\
\end{split}
\end{equation}
Putting them together yields the following results:
\begin{equation}
\mathbf{Vb}=\left [
\begin{array}{c}
\mathbf{v}_{12}^\top\\
\mathbf{v}_{13}^\top\\
\mathbf{v}_{23}^\top\\
(\mathbf{v}_{11}-\mathbf{v}_{22})^\top\\
(\mathbf{v}_{22}-\mathbf{v}_{33})^\top\\
\end{array}
\right ]\mathbf{b}=0
\end{equation}
As a result, we can solve the following optimization problem to get $\mathbf{B}$:
\begin{equation}
\begin{split}
\min_{\mathbf{b}}&\quad \|\mathbf{Vb}\|_2^2\\
s.t.&\quad \|\mathbf{b}\|=1
\end{split}
\end{equation}
which again can be solved by SVD. One caveat in this optimization is that the optimal $\mathbf{b}$ need to be converted into $\mathbf{B}$ up to scale:
\begin{equation}
\begin{split}
&\beta\mathbf{B}\\
=&\left [
\begin{array}{ccc}
b_1 & 0 & b_2 \\
0 & b_1 & b_3 \\
b_2 & b_3 & b_4
\end{array}
\right ]\\
=&\left [
\begin{array}{ccc}
\frac{\beta}{f^2} & 0 & -\frac{\beta x}{f^2} \\
0 & \frac{\beta}{f^2} & -\frac{\beta y}{f^2} \\
-\frac{\beta x}{f^2} & -\frac{\beta y}{f^2} & \beta\frac{x^2+y^2}{f^2} + \beta
\end{array}
\right ]\\
\end{split}
\end{equation}
The camera parameters then can be pulled directly from $\mathbf{b}$:
\begin{equation}
\begin{split}
x=&-\frac{b_2}{b_1}\\
y=&-\frac{b_3}{b_1}\\
\beta=&b_4-\frac{b_2^2+b_3^2}{b_1}\\
f=&\sqrt{\frac{\beta}{b_1}}=\frac{\sqrt{b_4b_1-(b_2^2+b_3^2)}}{|b_1|}
\end{split}
\end{equation}
Once $\mathbf{K}$ is known, $\mathbf{R}$ can be solved by computing $\mathbf{R}\propto\mathbf{K}^{-1}[\mathbf{m}_1,\mathbf{m}_2,\mathbf{m}_3]$, and using SVD to make sure $\mathbf{R}$ is orthogonal.

\subsection{Optimizing shape deformation}
Here we only consider scaling the shape in its principal axis. Specifically, we want to estimate a diagonal matrix $\mathbf{S}$:
\begin{equation}
\mathbf{S}=\left [
\begin{array}{ccc}
s_x & 0 & 0 \\
0 & s_y & 0 \\
0 & 0 & s_z
\end{array}
\right ]
\end{equation}
For each pair of correspondent points $\mathbf{p}_i$ and $\mathbf{q}_i$, we want to fit $\mathbf{S}$ such that:
\begin{equation}
\begin{split}
\mathbf{q}_i=&s_i\mathbf{K}(\mathbf{R}\mathbf{S}\mathbf{p}_i + \mathbf{t})\\
=&s_i\mathbf{K}(\mathbf{R}\mathbf{P}_i\left [
\begin{array}{c}
s_x \\
s_y \\
s_z
\end{array}
\right ] + \mathbf{t})
\end{split}
\end{equation}
Here $\mathbf{P}_i$ is a diagonal matrix:
\begin{equation}
\mathbf{P}_i=\left [
\begin{array}{ccc}
p_{i1} & 0 & 0\\
0 & p_{i2} & 0\\
0 & 0 & p_{i3}
\end{array}
\right ]
\end{equation}
Define $\mathbf{D}^i = \mathbf{KRP}_i\in\mathcal{R}^{3\times 3}$ and $\mathbf{o}=\mathbf{Kt}\in\mathcal{R}^{3}$, we get 2 linear equations for each pair of correspondence:
\begin{equation}
\begin{split}
&d^i_{11}s_x+d^i_{12}s_y+d^i_{13}s_z+o_1\\
=&q_{i1}(d^i_{31}s_x+d^i_{32}s_y+d^i_{33}s_z+o_3)\\
&d^i_{21}s_x+d^i_{22}s_y+d^i_{23}s_z+o_2\\
=&q_{i2}(d^i_{31}s_x+d^i_{32}s_y+d^i_{33}s_z+o_3)\\
\end{split}
\end{equation}
which can be formulated into a linear least square problem:
\begin{equation}
\begin{split}
\min_{\mathbf{s}\in\mathcal{R}^3}&\quad \|\mathbf{D}\mathbf{s}-\mathbf{O}\|_2^2
\end{split}
\end{equation}
Here $\mathbf{D}\in\mathcal{R}^{2n\times 3}$, $\mathbf{O}\in\mathcal{R}^{2n}$, and $\mathbf{s}=(\mathbf{D}^\top\mathbf{D})^{-1}\mathbf{D}^\top\mathbf{O}$.

%------------------------------------------------------------------------
\section{Experiments}

%------------------------------------------------------------------------
\section{Conclusion}

{\small
\bibliographystyle{ieee}
\bibliography{reportbib}
}

\end{document}
